\chapter[Referencial Teórico]{Referencial Teórico (Em andamento)}

Neste capitulo é apresentada as principais referencias para a pesquisa.

\section{Hesitação Vacinal}
Em 2019, a \cite{noauthor_ten_nodate} levantou dez ameaças à saúde global, entre as quais, faz parte, a hesitação vacinal. De acordo com a OMS, os principais motivos para essa hesitação são as dificuldades de acesso à vacina, complacência e falta de confiança. A mesma também afirma que os casos de sarampo cresceram em 30\% no mundo, inclusive em países que estavam perto de erradicá-la, devido à essa hesitação.

    \subsection{Hesitação Vacinal em Aplicações Web}
    
    \cite{puri_social_2020} afirma que um dos fatores que contribuíram para o crescimento da hesitação vacinal foi o surgimento das mídias sociais. Essas aplicações permitem que os usuários consumam e produzem conteúdo livremente sem a curadoria de uma editora ou autoridade no assunto abordado. Além disso, os algoritmos dessas plataformas apresentam para seus usuários conteúdos personalizados de acordo com as interações em postagens anteriores. Isso significa que se um indivíduo interage com contas ou conteúdos que desencorajam a vacinação, essas mídias continuarão a apresentar cada vez mais conteúdos desse tipo, inserindo essas pessoas em um ambiente que dificilmente mostrará um conteúdo divergente de suas opiniões ou crenças.
    
    Em 2017, ainda de acordo com \cite{puri_social_2020}, houve uma pesquisa com 87 vídeos do Youtube relacionados à segurança de vacinas e vacinação infantil. Entre eles, 65\% foram vídeos que desestimularam a vacinação, 36.8\% vídeos não continham embasamento científico e apenas 5.6\% eram oriundos de profissionais governamentais. Outras pesquisas revelaram que conteúdos anti vacina têm mais chance de serem engajados, recebendo mais curtidas e sendo mais compartilhados dentro das redes.
    
    \cite{puri_social_2020} cita outros estudos que demonstraram que visualizar conteúdo anti-vacina pode afetar diretamente quão seguro o usuário se sente em se vacinar. Esses estudos mostraram que é mais fácil desencorajar alguém a se vacinar apenas com estes conteúdos do que encorajá-la e, apesar de não ser claro o motivo, é provável que isso ocorra, pois os riscos de uma vacinação parecem mais reais e próximos quando comparados a um benefício mais a longo prazo do controle da doença.
    
    \cite{bragazzi_how_2017} realizou um estudo para descobrir a frequência com que as pessoas procuram informações sobre vacinas no Google. Os resultados dessa pesquisa foram:

    \begin{itemize}
        \item Quando pesquisando por doenças infecciosas, buscas relacionadas à vacina estão em terceiro lugar;
        \item Essas buscas são influenciadas pela mídia, ou seja, tendem a aumentar quando há maior cobertura pelas mídias, e;
        \item Os usuários normalmente buscam mais por efeitos colaterais;
    \end{itemize}

% \section{Vacinação no Brasil}

\section{Sistemas de Informação de Imunização (IIS)}
    \subsection{Sistemas de Informação de Imunização no Mundo}
    \cite{gianfredi_countering_2019} conduziu uma revisão sobre Sistemas de Informação de Imunização (IIS, do inglês, \textit{Immunization Information Systems}) para levantar as vantagens no combate à hesitação de vacina. Entre os recursos destes sistemas estão gerenciamento da logística, notificações de vacina e monitoramento de estoques e efeitos adversos. Vale ressaltar que um sistema não possui, obrigatoriamente, todos estes recursos.

    Essa revisão cita um estudo em que foram utilizados cartões postais, mensagens de texto, \textit{e-mail} e ligações como forma de lembrar sobre a vacinação, além de um grupo de controle que não recebeu nenhuma notificação. O resultado desse estudo mostrou que quem recebeu as mensagens de texto teve a maior taxa de vacinação, 32.1\% contra 9.7\% do grupo de controle e, que este mesmo grupo demorou menos tempo entre o recebimento da mensagem e o recebimento do imunizante.
    
    \cite{gianfredi_countering_2019} também traz uma pesquisa em que foram comparados os impactos de mensagens de texto educativas, educativas e interativas e ligações por telefone na vacinação de crianças contra gripe que não haviam sido imunizadas até o final de 2011. Os resultados dessa pesquisa confirmaram o efeito positivo das mensagens de texto, em que as mensagens que possuíam interação foram ainda mais eficazes. Uma provável explicação para este efeito foi a de que a interação gera um senso de responsabilidade maior nos pais devido ao engajamento. Outros estudos mostraram o uso de IIS para identificar as razões pelas quais as pessoas se recusam a se vacinar, permitindo, assim, a criação de campanhas mais efetivas.
    
    \cite{gianfredi_countering_2019} concluiu sua revisão afirmando que o uso de Sistemas de Informação de Imunização é útil para levantar a cobertura de vacinação, sua eficácia e segurança. Os sistemas que possuem notificação e feedback reduzem a hesitação e o tempo que a pessoa leva para receber o imunizante. E, por fim, podem auxiliar a melhorar o alcance e performance de campanhas.
    
    \subsection{Sistemas de Informação de Imunização no Brasil}
        
        \subsubsection{Programa nacional de imunização}
        
        
        \subsubsection{Conecte-SUS}
        Segundo \cite{harzheim_bases_2020}, o Conecte-SUS é uma ferramenta do ministério da saúde para integrar dados clínicos e administrativos da área de saúde através do CPF das pessoas que estão participando dele, este sistema cobre desde armazenamento e possibilidade de comprovação sobre quais imunizantes o indivíduo consumiu como grande parte de suas informações médicas disponíveis de forma segura e sigilosa.

        \subsubsection{Informatiza-APS}
        Ainda segundo \cite{harzheim_bases_2020}, para o Conecte-SUS funcionar este necessita que as informações armazenadas nos centros de saúde, postos de saúde ,hospitais públicos e outros estabelecimentos de saúde pública um alto grau de informatização e que utilizem de prontuários eletrônicos podendo ser o eSUS-AB ou outro prontuário eletrônico desde que entregue os dados de Saúde da Família via integrador \textit{thrift} para o Sistema de Informação em Saúde para a Atenção Básica (SISAB), porém não são todos os municípios que adotavam prontuários eletrônicos em meados de 2019 então foi criado um programa de incentivo ao uso de prontuário eletrônico chamado de Informatiza-APS.
